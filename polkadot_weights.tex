\documentclass[11pt,a4paper]{article}
\usepackage{geometry}
\geometry{
    a4paper,
    total={170mm,257mm},
    left=20mm,
    top=20mm,
}
\usepackage{color}
\usepackage{hyperref}
\usepackage{amsmath}
\newcommand{\todo}[1]{\textcolor{red}{TODO: #1}}
\newcommand{\SubItem}[1]{
    {\setlength\itemindent{15pt} \item[-] #1}
}

\begin{document}
\title{Polkadot Weights}
\author{Web3 Foundation}
\date{April 2020}
\maketitle

\section{Motivation}
Polkadot has a limited time window for block producers to create a block,
including limitations on block size which can make the selection and execution
of certain extrinsics too expensive and decelerate the network. The weight
system introduces a mechanism for block producers to measure the expense of
extrinsics and determine how "heavy" it is. With this mechanism, block producers
can select a set of extrinsics and saturate the block to it's fullest potential
without exceeding any limitations (as described in section \ref{sec:limitations}).
\newline

Polkadot also introduces a specified block ratio (as defined in section \ref{sec:limitations}),
ensuring that only a certain portion of the total block size gets used for regular extrinsics.
The remaining space is reserved for critical, operational extrinsics required for the functionality
by Polkadot itself.

\section{Fundamentals}
Weights are just a numeric value and Runtime functions may use complex structures to express those
values. Therefore, the following requirements must apply for implementing weight calculations:
\begin{itemize}
\item Computations of weights must be determined before execution of that extrinsic.
\item Due to the limited time window, computations of weights must be done quickly and consume
      few resources themselves.
\item Weights must be self contained and must not require I/O on the chain state. Weights are
      fixed measurements and are based solely on the Runtime function and its parameters.
\item Weights serve tree functions: measurements used to calculate transaction fees, to prevent
      the block being filled with too many extrinsics and to avoid extrinsics where its execution
      takes too long.
\end{itemize}

\section{Limitations}\label{sec:limitations}
The assigned weights should be relative to each others execution time and "heaviness",
although weights can be assigned depending on the priorities the chain is supposed to endorse.
Following limitations must be considered when assigning weights, which vary on the Runtime.

\subsection{Considerable limitations}
\begin{itemize}
\item Maximum block length (in bytes)
\item Maximum block weight
\item Targeted time per block
\item Available block ration reserved for normal, none-operational transactions
\end{itemize}

\subsection{Considerable limitations in Polkadot}
As of the official Polkadot Runtime, the limitations are set as follows:

\begin{itemize}
\item Maximum block length: $5 \times 1'024 \times 1'024 = 5'242'880$
\item Maximum block weight: 1'000'000'000
\item Targeted time per block: 6 seconds
\item Available block ratio: 75\%
\end{itemize}

The values of the assigned weight itself is not relevant. It must only fulfill the requirements
as noted by the fundamentals and limitations, and can be assigned as the author sees fit.
As a simple example, consider a maximum block weight of 1'000'000'000, an available ratio of
75\% and a targeted transaction throughput of 500 transactions, we could assign the weight
for each transaction at about 1'500'000.

\section{Weight Assignment}
Assigning weights based on theoretical performance such as big O notation proves to be
unreliable and too complex due to imprecision in back-end systems, internal communication
within the Runtime and design choices in the software. Therefore, all available Runtime 
functions, which create and execute extrinsics, have to be benchmarked with a large
collection of input parameters.

\subsection{Parameters}
The inputs parameters highly vary depending on the Runtime function and must therefore
be carefully selected. The benchmarks should use input parameters which will most likely be
used in regular cases, as intended by the authors, but must also consider worst case
scenarios and inputs which might decelerate or heavily impact performance of the function.
The input parameters should be randomized in order to cause various effects in behaviors
on certain values, such as memory relocations and other results that can impact performance.

\subsection{Blockchain State}
The benchmarks should be performed on blockchain states that already contain a history of
extrinsics and storage changes. Runtime functions that required read/writing on structures
such as Tries will therefore produce more realistic results that will reflect the real-world
performance of the Runtime.

\subsection{Environment}
The benchmarks should be executed on clean systems without interference of other processes
or software. Additionally, the benchmarks should be executed multiple machines with different
system resources, such as CPU performance, CPU cores, RAM and storage speed.
\newpage

\section{Fees}
Block producers charge a fee in order to be economically sustainable. That fee must always
be covered by the sender of the transaction. Polkadot has a flexible mechanism to determine
the minimum cost to include transactions in a block.

\subsection{Fee Calculation}
Polkadot fees consists of three parts:

\begin{itemize}
\item Base fee: a fixed fee that is applied to every transaction and set by the Runtime.
\item Length fee: a fee that gets multiplied by the length of the transaction, in bytes.
\item Weight fee: a fee for each, varying Runtime function. Runtime implementers need to
      implement a conversion mechanism which determines the corresponding currency amount
      for the calculated weight.
\end{itemize}

The final fee can be summarized as:
\begin{eqnarray*}
\lefteqn{fee = base\ fee}\\
      &&{} + length\ of\ transaction\ in\ bytes \times length\ fee\\
      &&{} + weight\ to\ fee\\
\end{eqnarray*}

\subsection{Definitions in Polkadot}
The Polkadot Runtime defines the following values:
\begin{itemize}
\item Base fee: 100 uDOTs
\item Length fee: 0.1 uDOTs
\item Weight to fee conversion:
      $$
            weight\ fee = weight \times (100\ uDOTs \div (10 \times 10'000\ uDOT))
      $$
      A weight of 10'000 (the smallest non-zero weight) is mapped to $\frac{1}{10}$ 100 uDOT.
      \newline
      This fee will never exceed $1.701411835 \times 10^{32}$ uDOTs.
\end{itemize}

\subsection{Fee Multiplier}
Polkadot can add a additional fee to transactions if the network becomes too busy and starts to
decelerate the system. This fees can create incentive to avoid the production of
low priority or insignificant transactions. In contrast, those additional fees will decrease if
the network calms down and it can execute transactions without much difficulties.
\newline

That additional fee is known as the \verb|Fee Multiplier| and its value is defined
by the Polkadot Runtime. The multiplier works by comparing the saturation of blocks; if the previous 
block is less saturated than the current block (implying an uptrend), the fee is slightly increased.
Similarly, if the previous block is more saturated than the current block (implying a downtrend), the
fee is slightly decreased.
\newline

The final fee is calculated as:
\begin{verbatim}
      final fee = fee * Fee Multiplier
\end{verbatim}
\newpage

\subsubsection{Update Multiplier}
The \verb|Update Multiplier| defines how the multiplier can change. The Polkadot Runtime internally
updates the multiplier according the following formula:

\begin{verbatim}
diff = (target_weight - previous_block_weight)
v = 0.00004
next_weight = weight * (1 + (v . diff) + (v . diff)^2 / 2)
\end{verbatim}

Polkadot defines the \verb|target_weight| as 0.25 (25\%). More information about this algorithm is described
in the Web3 Foundation research paper: \url{https://research.web3.foundation/en/latest/polkadot/Token Economics/#relay-chain-transaction-fees}.

\end{document}